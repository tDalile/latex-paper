\documentclass[a4paper, 10pt]{article}

% ============================================  USERSETTINGS  ==========================================

\newcommand{\myTitle}{Small \LaTeX{} Article Template\xspace}
\newcommand{\mySubtitle}{Subtitle\xspace}
\newcommand{\myDegree}{Degree\xspace}
\newcommand{\myName}{First Name Last Name\xspace}
\newcommand{\myProf}{Prof. Name\xspace}
\newcommand{\myOtherProf}{Prof. Name\xspace}
\newcommand{\mySupervisor}{Supervisor Name\xspace}
\newcommand{\myFaculty}{Faculty\xspace}
\newcommand{\myDepartment}{Department\xspace}
\newcommand{\myUni}{University\xspace}
\newcommand{\myLocation}{Location, Country\xspace}
\newcommand{\myTime}{\today\xspace}
\newcommand{\myVersion}{version 1.0\xspace}

% === USERSETTINGS FORMATTING: TITLE === 
\title{\vspace{-15mm}\fontsize{24pt}{10pt}\selectfont\textbf{\myTitle}} % Article title

\author{
\large
\textsc{\myName}\\[2mm]
\normalsize \myUni \\
\normalsize \myFaculty \\ 
\normalsize \href{mailto:test@test.com}{test@test.com} % change to your mail
\vspace{-5mm}
}
\date{}


% ============================================  PACKAGES  ==============================================

% === TEMPLATE RELATED: Remove in case of usage! ===
\usepackage{lipsum} % Package to generate dummy text throughout this template

% === %unsorted% ===
\usepackage{filecontents}


% === FONT & LANGUAGE ===
% \usepackage[ngerman]{babel} % set language
\usepackage[main=english,ngerman]{babel} % set multi-language document

\usepackage{microtype} % Slightly tweak font spacing for aesthetics
\usepackage[utf8]{inputenc}
\usepackage[T1]{fontenc} % needed if Umlauts are used
\usepackage{xspace} % adds a space unless the macro is followed by certain punctuationcharacters

% === MULTICOLUMN ===
\usepackage{multicol} 

% === Abstract ===
\usepackage{abstract} % Allows abstract customization
\renewcommand{\abstractnamefont}{\normalfont\bfseries} % Set the "Abstract" text to bold
\renewcommand{\abstracttextfont}{\normalfont\small\itshape} % Set the abstract itself to small italic text

% === SECTIONS ===
\renewcommand\thesection{\Roman{section}} % sections with roman numerals
\renewcommand\thesubsection{\Roman{subsection}} % subsections with roman numerals
%\titleformat{\section}[block]{\large\scshape\centering}{\thesection.}{1em}{} % new style of section titles
%\titleformat{\subsection}[block]{\large}{\thesubsection.}{1em}{} % new style of section titles

% === HEADERS & FOOTERS
%\usepackage{fancyhdr} 
%\pagestyle{fancy} % set for all pages
%\fancyhead{} % Blank out the default header
%\fancyfoot{} % Blank out the default footer
%\fancyhead[C]{Running title $\bullet$ November 2012 $\bullet$ Vol. XXI, No. 1} % Custom header text
%\fancyfoot[RO,LE]{\thepage} % Custom footer text

% === FIGURES ===
\usepackage{graphicx}

% === TABLES ===
\usepackage{booktabs}  % Horizontal rules in tables

% === Algorithms ===
\usepackage{algorithm}
\usepackage{algorithmic}

% === Hyperlinks ===
\usepackage[backref=page,hidelinks]{hyperref}

% === FIXES ===
\newcommand{\theHalgorithm}{\arabic{algorithm}} % hyperref and algorithmic misbehave sometimes
\usepackage{float} % Required for tables and figures in the multi-column environment - they need to be placed in specific locations with the [H] (e.g. \begin{table}[H])
\usepackage{paralist} % less space between bullet points

% ============================================  FINETUNING  =============================================

\renewcommand*{\backref}[1]{}
%\renewcommand*{\backrefalt}[4]{{\footnotesize [%
%    \ifcase #1 Not cited.%
%    \or Cited on page~#2%
%    \else Cited on pages #2%
%    \fi%
%]}